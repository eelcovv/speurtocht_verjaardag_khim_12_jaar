\documentclass{article}
\title{Speurtocht Driemanspolder 2024}
\author{Eelco van Vliet}
\date{8 juni 2024}
\pagestyle{empty}
\usepackage[dutch]{babel}
\usepackage[top=30pt,bottom=30pt,left=48pt,right=46pt]{geometry}
\usepackage{hyperref}
\hypersetup{pdfborder = 0 0 0}
\usepackage{graphicx}
\usepackage{tabularx}
\usepackage{booktabs}
\usepackage[skip=1ex, figurename=Vraag]{caption}
% Letters form word: MARSHMALLOWSFEESTJE!
% VRAAG 1 Jupiter
\newcommand{\locone}{52° 3'36N 4°26'00E}
\newcommand{\urlone}{https://forms.gle/88Jf8wcjFSU8aojt8}
\newcommand{\letterone}{A}
\newcommand{\noteone}{Jupiter}
% VRAAG 2 Honkbalknuppel
\newcommand{\loctwo}{52° 3'45"N 4°26'11"E}
\newcommand{\urltwo}{https://forms.gle/jwE6WMcVdYzMEdi86}
\newcommand{\lettertwo}{S}
\newcommand{\notetwo}{Honkbalknuppel}
% VRAAG 3 Irish Kids
\newcommand{\locthree}{52° 3'51"N 4°26'24"E}
\newcommand{\urlthree}{https://forms.gle/kxh8LWZneB8L2Z3z6}
\newcommand{\letterthree}{W}
\newcommand{\notethree}{Irish Kids}
% VRAAG 4 Heliumballon
\newcommand{\locfour}{52° 3'59"N 4°26'31"E}
\newcommand{\urlfour}{https://forms.gle/V9HNHxqCGrrNP5G88}
\newcommand{\letterfour}{R}
\newcommand{\notefour}{Heliumballon}
% VRAAG 5
\newcommand{\locfive}{52° 4'1"N 4°26'28"E}
\newcommand{\urlfive}{https://example.com}
\newcommand{\letterfive}{M}
\newcommand{\notefive}{}
% VRAAG 6
\newcommand{\locsix}{52° 4'6"N 4°26'4"E}
\newcommand{\urlsix}{https://example.com}
\newcommand{\lettersix}{F}
\newcommand{\notesix}{}
% VRAAG 7
\newcommand{\locseven}{52° 3'58"N   4°25'42"E}
\newcommand{\urlseven}{https://example.com}
\newcommand{\letterseven}{M}
\newcommand{\noteseven}{}
% VRAAG 8
\newcommand{\loceight}{52° 4'10"N   4°25'23"E}
\newcommand{\urleight}{https://example.com}
\newcommand{\lettereight}{L}
\newcommand{\noteeight}{}
% VRAAG 9
\newcommand{\locnine}{52° 4'12"N  4°25'14"E}
\newcommand{\urlnine}{https://example.com}
\newcommand{\letternine}{O}
\newcommand{\notenine}{}
% VRAAG 10
\newcommand{\locten}{52° 4'15"N   4°25'7"E}
\newcommand{\urlten}{https://example.com}
\newcommand{\letterten}{E}
\newcommand{\noteten}{}
% VRAAG 11
\newcommand{\loceleven}{52° 4'11"N   4°25'6"E}
\newcommand{\urleleven}{https://example.com}
\newcommand{\lettereleven}{S}
\newcommand{\noteeleven}{}
% VRAAG 12
\newcommand{\loctwelve}{52° 4'10"N   4°25'1"E}
\newcommand{\urltwelve}{https://example.com}
\newcommand{\lettertwelve}{A}
\newcommand{\notetwelve}{}
% VRAAG 13
\newcommand{\locthirteen}{52° 3'53"N   4°25'17"E}
\newcommand{\urlthirteen}{https://example.com}
\newcommand{\letterthirteen}{E}
\newcommand{\notethirteen}{}
% VRAAG 14
\newcommand{\locfourteen}{52° 3'47"N   4°25'22"E}
\newcommand{\urlfourteen}{https://example.com}
\newcommand{\letterfourteen}{J}
\newcommand{\notefourteen}{}
% VRAAG 15
\newcommand{\locfifteen}{52° 3'41"N   4°25'26"E}
\newcommand{\urlfifteen}{https://example.com}
\newcommand{\letterfifteen}{T}
\newcommand{\notefifteen}{}
% VRAAG 16
\newcommand{\locsixteen}{52° 3'37"N   4°25'43"E}
\newcommand{\urlsixteen}{https://example.com}
\newcommand{\lettersixteen}{E}
\newcommand{\notesixteen}{}
% VRAAG 17
\newcommand{\locseventeen}{52° 3'40"N   4°25'44"O}
\newcommand{\urlseventeen}{https://example.com}
\newcommand{\letterseventeen}{H}
\newcommand{\noteseventeen}{}
% VRAAG 18
\newcommand{\loceighteen}{52° 3'41"N   4°25'56"E}
\newcommand{\urleighteen}{https://example.com}
\newcommand{\lettereighteen}{L}
\newcommand{\noteeighteen}{}
% VRAAG 19
\newcommand{\locnineteen}{ 52° 3'37"N  4°25'59"E}
\newcommand{\urlnineteen}{https://example.com}
\newcommand{\letternineteen}{S}
\newcommand{\notenineteen}{}


\begin{document}
    \maketitle
        \begin{tabular}{llll}
            \toprule
            \multicolumn{4}{l}{Het te raden woord is: \emph{MARSHMALLOWSFEESTJE!}} \\
            \textbf{Vraag} & \textbf{Letter} & \textbf{Locatie} & \textbf{Omschrijving} \\
            \midrule
            % VRAAG 1
             \ref{fig:question1} & \letterone & \href{\urlone}{\locone} & \noteone \\
%            % VRAAG 2
            \ref{fig:question2} & \lettertwo & \href{\urltwo}{\loctwo} & \notetwo \\
%            % VRAAG 3
            \ref{fig:question3} & \letterthree & \href{\urlthree}{\locthree} & \notethree  \\
%            % VRAAG 4
            \ref{fig:question4} & \letterfour  & \href{\urlfour}{\locfour} & \notefour \\
%            % VRAAG 5
            \ref{fig:question5} & \letterfive  & \href{\urlfive}{\locfive} & \notefive \\
%            % VRAAG 6
            \ref{fig:question6} & \lettersix  & \href{\urlsix}{\locsix} & \notesix \\
%            % VRAAG 7
            \ref{fig:question7} & \letterseven  & \href{\urlseven}{\locseven} & \noteseven \\
%            % VRAAG 8
            \ref{fig:question8} & \lettereight  & \href{\urleight}{\loceight} & \noteeight \\
%            % VRAAG 9
            \ref{fig:question9} & \letternine  & \href{\urlnine}{\locnine} & \notenine \\
%            % VRAAG 10
            \ref{fig:question10} & \letterten  & \href{\urlten}{\locten} & \noteten \\
%            % VRAAG 11
            \ref{fig:question11} & \lettereleven  & \href{\urleleven}{\loceleven} & \noteeleven \\
%            % VRAAG 12
            \ref{fig:question12} & \lettertwelve  & \href{\urltwelve}{\loctwelve} & \notetwelve \\
%            % VRAAG 13
            \ref{fig:question13} & \letterthirteen  & \href{\urlthirteen}{\locthirteen} & \notethirteen \\
%            % VRAAG 14
            \ref{fig:question14} & \letterfourteen  & \href{\urlfourteen}{\locfourteen} & \notefourteen \\
%            % VRAAG 15
            \ref{fig:question15} & \letterfifteen  & \href{\urlfifteen}{\locfifteen} & \notefifteen \\
%            % VRAAG 16
            \ref{fig:question16} & \lettersixteen  & \href{\urlsixteen}{\locsixteen} & \notesixteen \\
%            % VRAAG 17
            \ref{fig:question17} & \letterseventeen  & \href{\urlseventeen}{\locseventeen} & \noteseventeen \\
%            % VRAAG 18
            \ref{fig:question18} & \lettereighteen  & \href{\urleighteen}{\loceighteen} & \noteeighteen \\
%            % VRAAG 19
            \ref{fig:question19} & \letternineteen  & \href{\urlnineteen}{\locnineteen} & \notenineteen \\
            \bottomrule
        \end{tabular}\label{tab:table}


    \begin{tabularx}{\columnwidth}{XXX}
        \includegraphics[width=\linewidth]{figures/qr_vraag_1}
        \captionof{figure}{\locone}\label{fig:question1}
        &
        \includegraphics[width=\linewidth]{figures/qr_vraag_2}
        \captionof{figure}{\loctwo}\label{fig:question2}
        &
        \includegraphics[width=\linewidth]{figures/qr_vraag_3}
        \captionof{figure}{\locthree}\label{fig:question3}
        \\
        \includegraphics[width=\linewidth]{figures/qr_vraag_4}
        \captionof{figure}{\locfour}\label{fig:question4}
        &
        \includegraphics[width=\linewidth]{example-image-b}
        \captionof{figure}{\locfive}\label{fig:question5}
        &
        \includegraphics[width=\linewidth]{example-image-c}
        \captionof{figure}{\locsix}\label{fig:question6}
        \\
        \includegraphics[width=\linewidth]{example-image-a}
        \captionof{figure}{\locseven}\label{fig:question7}
        &
        \includegraphics[width=\linewidth]{example-image-a}
        \captionof{figure}{\loceight}\label{fig:question8}
        &
        \includegraphics[width=\linewidth]{example-image-a}
        \captionof{figure}{\locnine}\label{fig:question9}
        \\
        \includegraphics[width=\linewidth]{example-image-a}
        \captionof{figure}{\locten}\label{fig:question10}
        &
        \includegraphics[width=\linewidth]{example-image-a}
        \captionof{figure}{\loceleven}\label{fig:question11}
        &
        \includegraphics[width=\linewidth]{example-image-a}
        \captionof{figure}{\loctwelve}\label{fig:question12}
        \\
    \end{tabularx}

    \clearpage
    \begin{tabularx}{\columnwidth}{XXX}
        \includegraphics[width=\linewidth]{example-image-a}
        \captionof{figure}{\locthirteen}\label{fig:question13}
        &
        \includegraphics[width=\linewidth]{example-image-a}
        \captionof{figure}{\locfourteen}\label{fig:question14}
        &
        \includegraphics[width=\linewidth]{example-image-a}
        \captionof{figure}{\locfifteen}\label{fig:question15}
        \\
        \includegraphics[width=\linewidth]{example-image-a}
        \captionof{figure}{\locsixteen}\label{fig:question16}
        &
        \includegraphics[width=\linewidth]{example-image-a}
        \captionof{figure}{\locseventeen}\label{fig:question17}
        &
        \includegraphics[width=\linewidth]{example-image-a}
        \captionof{figure}{\loceighteen}\label{fig:question18}
        \\
        \includegraphics[width=\linewidth]{example-image-a}
        \captionof{figure}{\locnineteen}\label{fig:question19}
        &
        &
        \\
    \end{tabularx}
\end{document}